%! BibTeX Compiler = biber
%TC:ignore
\documentclass[10pt]{article}
\usepackage{geometry}
\geometry{a4paper, margin=1in}
\usepackage{xcolor}
\definecolor{BLUELINK}{HTML}{0645AD}
\definecolor{DARKBLUELINK}{HTML}{0B0080}
\definecolor{LIGHTGREY}{gray}{0.9}
\PassOptionsToPackage{hyphens}{url}
\usepackage[colorlinks=false]{hyperref}
% for linking between references, figures, TOC, etc in the pdf document
\hypersetup{colorlinks,
    linkcolor=DARKBLUELINK,
    anchorcolor=DARKBLUELINK,
    citecolor=DARKBLUELINK,
    filecolor=DARKBLUELINK,
    menucolor=DARKBLUELINK,
    urlcolor=BLUELINK
} % Color citation links in purple
\PassOptionsToPackage{unicode}{hyperref}
\PassOptionsToPackage{naturalnames}{hyperref}

\usepackage[backend=biber,eprint=false,isbn=false,url=false,intitle=true,style=nature,date=year]{biblatex}
\addbibresource{references.bib}

\usepackage{amssymb,amsfonts,amsmath,amsthm,mathtools}
\usepackage{lmodern}
\usepackage{xfrac, nicefrac}
\usepackage{bm}
\usepackage{listings, enumerate, enumitem}
\usepackage[export]{adjustbox}
\usepackage{graphicx}
\usepackage{bbold}
\usepackage{pdfpages}
\pdfinclusioncopyfonts=1

\renewcommand{\baselinestretch}{1.5}
\renewcommand{\arraystretch}{0.6}
\frenchspacing

\newcommand{\UniDimArray}[1]{\bm{#1}}
\newcommand{\der}{\text{d}}
\newcommand{\e}{\text{e}}
\newcommand{\Ne}{N_{\text{e}}}
\newcommand{\dn}{d_{\text{N}}}
\newcommand{\ds}{d_{\text{S}}}
\newcommand{\dnds}{\dn / \ds}
\newcommand{\pin}{\pi_{\text{N}}}
\newcommand{\pis}{\pi_{\text{S}}}
\newcommand{\pinpis}{\pin / \pis}
\newcommand{\piInter}{\pi_{\text{I}}}

\begin{document}

    \part*{Rate of evolution as function of expression level and $\Ne$}
    \tableofcontents
    \clearpage


    \section{Theoretical expectation}

    Thermodynamic equations allow one to derive the proportion of protein molecules that are in the native (folded) conformation in the cytoplasm.
    We assume that each misfolded protein molecule has the same selective cost, caused by its toxicity for the cell.
    Under this model, the total selective cost of a destabilizing mutation is now directly proportional to the total amount of misfolded proteins, and is proportional the expression level $y$.
    It is then possible to derive the equilibrium at mutation-selection equilibrium as a function of expression level $y$.
    Moreover, under different effective population size ($\Ne$), the strength of selection exerted on destabilizing mutations is different, and thus the equilibrium is different.

    At the specific equilibrium between mutation, selection and drift, the rate of evolution is given by the probability of fixation of a selected mutation (relative to neutral mutation), called $\omega$.
    In practice, $\omega$ is a approximated from the ratio of non-synonymous to synonymous polymorphism, $\pinpis$, and or divergence, $\dnds$.
    Altogether, it is possible to derive analytically the change in $\omega$ as a function of $\Ne$ and $y$ as in \textcite[eq.~18]{latrille_quantifying_2021}.
    \begin{align}
        \chi = \frac{ \Delta \omega}{\Delta \log (\Ne)} = \frac{ \Delta \omega}{\Delta \log (y)} = C < 0,
    \end{align}
    where $C$ is a constant that depends on thermodynamic parameters.
    From this equation, $\omega$ is linearly decreasing with $\Ne$ (in log scale) as well as with $y$ (in log scale), importantly the slope of the linear model is the same for both.


    \section{Computating genetic diversity and divergence}

    \subsection{Number of sites}\label{subsec:nunber-of-sites}
    Between different genes, polymorphism and divergence counts are not directly comparable because they are not in the same unit and are mechanically higher for genes with more sites.
    Moreover, even under neutrality, non-synonymous polymorphism counts are expected higher than synonymous counts because a mutation is more likely to be non-synonymous than synonymous.
    This argument is also true for non-synonymous and synonymous substitutions, respectively, that must be corrected for.

    Deriving the number of sites is thus necessary to correct polymorphism and divergence count such as to obtain non-synonymous divergence ($\dn$), synonymous divergence ($\ds$), synonymous polymorphism ($\pin$) and non-synonymous polymorphism ($\pis$) in the same unit.
    For each gene, all possible nucleotide mutations were computed from the reference protein-coding DNA sequence ($3 \times L$ mutations for a sequence of $L$ nucleotides).
    Whether a mutation was synonymous or non-synonymous was determined by comparing the reference codon to the codon obtained after the mutation.
    Moreover, each mutation was weighted by the instantaneous rate of change between nucleotides, derived from fitting a nucleotide substitution model to the b10k alignment.
    The proportion of synonymous mutations was then given as the sum of the instantaneous rates of all synonymous mutations, divided by the sum across all possible mutations (synonymous, non-synonymous, stop).
    This proportion of mutations being synonymous is multiplied by the number of sites in the gene to obtain the number of synonymous sites.
    Repeating this process for non-synonymous mutations gives the number of non-synonymous sites.

    \subsection{Genetic diversity ($\pi$, $\pin$, $\pis$)}
    As an estimator of genetic diversity, we used Tajima's $\pi$, the average pairwise difference between all sequences in the sample.
    $\pi$ was obtained for each population from the site-frequency spectrum (SFS) as in~\textcite[eq.~5-6]{achaz_frequency_2009}.
    Formally, $\UniDimArray{\xi}$ is a vector that represents the unfolded frequency spectrum composed of $\xi_i$, the number of polymorphic sites at frequency $i/n$ in the sample ($1 \leq i \leq n - 1$), where $n=48$ is the sample size (twice the number of individuals) in the population.
    $\pi$ is a function of $\UniDimArray{\xi}$ as:
    \begin{align}
        \pi (\UniDimArray{\xi}) =  \frac{\sum_{i=1}^{n - 1} (n - i)\times i \times \xi_i}{\sum_{i=1}^{n - 1} (n - i)},
    \end{align}
    $\pi$ was computed separately for non-synonymous ($\pin$), synonymous ($\pis$) polymorphism and also for inter-genic regions ($\piInter$, see below).
    Finally, to correct $\dn$, $\ds$, $\pin$, $\pis$ such that they are comparable between them, they are expressed in the same unit (per site) by normalizing with the number of non-synonymous or synonymous sites (see above), respectively.


    \section{$\omega$ as function of expression level}

    \subsection{Estimation}
    $\omega$ represents the rate of evolution of a protein, either estimated from polymorphism ($\pinpis$) or divergence ($\dnds$).
    $\omega$ was computed for each gene as the ratio of non-synonymous to synonymous polymorphism ($\pinpis$) or divergence ($\dnds$).
    To compute $\pinpis$ or $\dnds$, each gene must have at least one synonymous count, otherwise the ratio is undefined.
    $\chi$ is the slope of the linear regression of $\omega$ as a function of $\log(y)$, where $y$ is the expression level of the gene in TPM (transcripts per million).
    We computed $\chi$ independently in the two penguins populations (Emperor, E) and (King, K), and we denote $\chi^{\text{E}}$ and $\chi^{\text{K}}$ their estimates of $\chi$ respectively.

    To assess the robustness of the results and assess the fit of the linear model, we performed the same analysis while binning genes by their expression level.
    We performed the analysis with respectively 10, 25, 50 and 100 bins, and computed the slope of the linear regression $\chi$ and R$^2$.

    \subsubsection{Results with no bins}
    \begin{center}
        \includegraphics[width=0.85\textwidth]{../results/div_eLevel/0bins_0cutoff}
        \includegraphics[width=0.85\textwidth]{../results/poly_eLevel/0bins_0cutoff}
    \end{center}

    \subsubsection{Results with 100 bins}
    \begin{center}
        \includegraphics[width=0.85\textwidth]{../results/div_eLevel/100bins_0cutoff}
        \includegraphics[width=0.85\textwidth]{../results/poly_eLevel/100bins_0cutoff}
    \end{center}

    \subsubsection{Results with 50 bins}
    \begin{center}
        \includegraphics[width=0.85\textwidth]{../results/div_eLevel/50bins_0cutoff}
        \includegraphics[width=0.85\textwidth]{../results/poly_eLevel/50bins_0cutoff}
    \end{center}

    \subsubsection{Results with 25 bins}
    \begin{center}
        \includegraphics[width=0.85\textwidth]{../results/div_eLevel/20bins_0cutoff}
        \includegraphics[width=0.85\textwidth]{../results/poly_eLevel/20bins_0cutoff}
    \end{center}

    The slope of $\pinpis$ as a function of log expression level is not dependent on the number of bins used to compute $\pinpis$ or $\dnds$.
    However, for few bins, linear model is a strong fit (high R$^2$), but the fit decreases as the number of bins increases.

    \section{$\omega$ as function of $\Ne$}

    \subsection{Estimation}
    Given the two penguins populations (Emperor, E) and (King, K), we also estimated $\chi$ as the change of $\omega$ as a function of $\Ne$:
    \begin{align}
        \chi & = \frac{\Delta \omega}{\Delta \log \left( \Ne \right)} \\
             & = \frac{\omega^{\text{E}} - \omega^{\text{K}}}{\log\left( \Ne^{\text{E}}\right) - \log \left(\Ne^{\text{K}}\right)}.
    \end{align}
    Under the assumption that the mutation rate ($u$) is the same between the two populations, and since $\pi = 4 \Ne u$ from neutral markers, $\chi$ simplifies to
    \begin{align}
        \chi & = \frac{\omega^{\text{E}} - \omega^{\text{K}}}{\log\left( \piInter^{\text{E}} / \piInter^{\text{K}} \right)},
    \end{align}
    where $\piInter^{\text{E}}$ and $\piInter^{\text{K}}$ are estimated from the inter-genic regions, which are assumed to be neutral.
    Here normalization by the number of sites is not required since $\pi^{\text{E}}$ and $\pi^{\text{K}}$ are already expressed in the same unit in the two populations (the same reference genome was used), which cancels out in the ratio $\piInter^{\text{E}} / \piInter^{\text{K}}$.
    $\omega$ (either $\pinpis$ or $\dnds$) is computed as the total count (polymorphism or divergence) across all genes, divided by the total number of sites across all genes, respectively for polymorphism and divergence.
    We performed a bootstrap sampling (1000 replicates) to estimate the confidence interval of $\chi$, where gene were sampled with replacement in each replicate.

    \subsubsection{Results}
    \begin{center}
        \includegraphics[width=0.85\textwidth]{../results/rate_diversity/0bins_0cutoff}
    \end{center}

    \section{Intrepretation of results}
    Under the assumption that misfolded proteins are toxic for the cell, destabilizing mutations are deleterious and are thus under purifying selection.
    Because the total amount of misfolded proteins would increase with the gene expression level ($y$), so is the strength of selection exerted on destabilizing mutations.
    Moreover, and independently, because efficacy of selection increases with the effective population size ($\Ne$), so is the strength of selection exerted on destabilizing mutations.
    This reasoning can be formalized, \textcite[eq.~18]{latrille_quantifying_2021}, predicting that the rate of non-synonymous over synonymous substitutions ($\dnds$) is linearly decreasing with $\log(y)$ and $\log(\Ne)$, and the slope of the linear model is the same for both.
    In this study, using polymorphism within species ($\pinpis$ instead of $\dnds$), we showed that the two estimates of the slope are negative are statistically different from zero.
    However, the estimated are not significantly different from each other, and the confidence intervals overlap.
    This result suggests that both the expression level ($y$) of the gene and the effective population size ($\Ne$) have a similar effect the strength of selection exerted within protein coding genes.
    Altogether, although the effect $y$ and $\Ne$ on protein evolution are usually considered separately, our results suggest that instead their effect should be considered together in integrated models of evolution.
    However, the result must be interpreted with caution since the theoretical expectation were derived with $\pinpis$ instead of $\dnds$, and stability of the results must be assessed more thoroughly with more populations.

    \printbibliography

\end{document}